% This is the Reed College LaTeX thesis template. Most of the work
% for the document class was done by Sam Noble (SN), as well as this
% template. Later comments etc. by Ben Salzberg (BTS). Additional
% restructuring and APA support by Jess Youngberg (JY).
% Your comments and suggestions are more than welcome; please email
% them to cus@reed.edu
%
% See http://web.reed.edu/cis/help/latex.html for help. There are a
% great bunch of help pages there, with notes on
% getting started, bibtex, etc. Go there and read it if you're not
% already familiar with LaTeX.
%
% Any line that starts with a percent symbol is a comment.
% They won't show up in the document, and are useful for notes
% to yourself and explaining commands.
% Commenting also removes a line from the document;
% very handy for troubleshooting problems. -BTS

% As far as I know, this follows the requirements laid out in
% the 2002-2003 Senior Handbook. Ask a librarian to check the
% document before binding. -SN

%%
%% Preamble
%%
% \documentclass{<something>} must begin each LaTeX document
\documentclass[12pt,twoside]{reedthesis}
% Packages are extensions to the basic LaTeX functions. Whatever you
% want to typeset, there is probably a package out there for it.
% Chemistry (chemtex), screenplays, you name it.
% Check out CTAN to see: http://www.ctan.org/
%%
\usepackage{graphicx,latexsym}
\usepackage{amssymb,amsthm,amsmath}
\usepackage{longtable,booktabs,setspace}
\usepackage{chemarr} %% Useful for one reaction arrow, useless if you're not a chem major
\usepackage[hyphens]{url}
\usepackage{rotating}
\usepackage[square,numbers]{natbib}
\usepackage{amsthm}
\usepackage{amsmath}
\usepackage{amssymb}
\usepackage{caption}
\usepackage{ textcomp }
\usepackage{subcaption}
\usepackage{upgreek}
\usepackage[table,xcdraw]{xcolor}
\usepackage{enumitem}
\usepackage{pifont}
\usepackage{listings}
\usepackage{algorithm}
\usepackage{algpseudocode}
\usepackage{hyperref}
\lstset{
    breaklines=true,
    tabsize=2,
    basicstyle=\ttfamily,
    literate={\ \ }{{\ }}1
}

\newlist{todolist}{itemize}{2}
\setlist[todolist]{label=$\square$}
\newcommand{\cmark}{\ding{51}}
\newcommand{\xmark}{\ding{55}}
\newcommand{\done}{\rlap{$\square$}{\raisebox{2pt}{\large\hspace{1pt}\cmark}}
\hspace{-2.5pt}}

%% \newtheorem{theorem}{Theorem}

%% \theoremstyle{definition}
%% \newtheorem{definition}[theorem]{Definition}

%% \makeatletter
%% \newcommand{\defineterm}[2]{%
%%   \@ifdefinable{#1}{\gdef#1{#2}\emph{#2}}}
%% \makeatother

% Comment out the natbib line above and uncomment the following two lines to use the new
% biblatex-chicago style, for Chicago A. Also make some changes at the end where the
% bibliography is included.
%\usepackage{biblatex-chicago}
%\bibliography{thesis}

% \usepackage{times} % other fonts are available like times, bookman, charter, palatino

\title{TITLE}
\author{Thomas Ulmer}
% The month and year that you submit your FINAL draft TO THE LIBRARY (May or December)
\date{May 2024}
\division{Mathematical and Natural Sciences}
\advisor{Dylan McNamee}
%If you have two advisors for some reason, you can use the following
\altadvisor{David Meyer}
%%% Remember to use the correct department!
\department{Computer Science}
% if you're writing a thesis in an interdisciplinary major,
% uncomment the line below and change the text as appropriate.
% check the Senior Handbook if unsure.
\thedivisionof{The Established Interdisciplinary Committee for Mathematics-Computer Science}
% if you want the approval page to say "Approved for the Committee",
% uncomment the next line
%\approvedforthe{Committee}

\setlength{\parskip}{0pt}
%%
%% End Preamble
%%
%% The fun begins:
\begin{document}

  \maketitle
  \frontmatter % this stuff will be roman-numbered
  \pagestyle{empty} % this removes page numbers from the frontmatter

% Acknowledgements (Acceptable American spelling) are optional
% So are Acknowledgments (proper English spelling)
    \chapter*{Acknowledgements}


% The preface is optional
% To remove it, comment it out or delete it.
    %\chapter*{Preface}
        %This is an example of a thesis setup to use the reed thesis document class.



    \chapter*{List of Abbreviations}


    \tableofcontents
% if you want a list of tables, optional
   % \listoftables
% if you want a list of figures, also optional
    \listoffigures

% The abstract is not required if you're writing a creative thesis (but aren't they all?)
% If your abstract is longer than a page, there may be a formatting issue.
    \chapter*{Abstract}
        In this thesis I consider the automatic generation of hedcut style portraits using photographs of faces. These renderings mimic those that appear in the Wall Street Journal which are produced by hand with pen and ink. I begin this work by following Kim et al. who render hedcuts by placing stipples so they follow not only the outline of facial features but also isophotes (lines with constant illumination). I apply a variety of image processing techniques to extract facial components and inform stipple placement and size to illustrate tone and depth of the face. I then expand on the work of Kim et al. by introducing an interactive tool that allows for a fully parameterizable version of their approach. Finally I experiment with allowing for areas of negative space and areas of stipple density variation. In this thesis I report on this tool and assess the quality of these methods.

        %\chapter*{Dedication}
        %You can have a dedication here if you wish.

  \mainmatter % here the regular arabic numbering starts
  \pagestyle{fancyplain} % turns page numbering back on

%The \introduction command is provided as a convenience.
%if you want special chapter formatting, you'll probably want to avoid using it altogether

    \chapter*{Introduction}
         \addcontentsline{toc}{chapter}{Introduction}
        \chaptermark{Introduction}
        \markboth{Introduction}{Introduction}
        % The three lines above are to make sure that the headers are right, that the intro gets included in the table of contents, and that it doesn't get numbered 1 so that chapter one is 1.
\chapter{Background}
\section{Representing Images on a Computer}
\chapter{Extracting Feature Lines}
\chapter{Stippling}
\chapter{Algorithmic Assessment}
\chapter{RGB to CIEL*a*b Color Conversion}
\chapter{pseudocode}
%\chapter{Lower Envelope Algorithm}
\section{Lower Envelope Algorithm}
\chapter{Results}
\section{Level 1 Input Images}

   \clearpage


  \backmatter % backmatter makes the index and bibliography appear properly in the t.o.c...

% if you're using bibtex, the next line forces every entry in the bibtex file to be included
% in your bibliography, regardless of whether or not you've cited it in the thesis.
    \nocite{*}

% Rename my bibliography to be called "Works Cited" and not "References" or ``Bibliography''
\renewcommand{\bibname}{Works Cited}

%    \bibliographystyle{bsts/mla-good} % there are a variety of styles available;
%  \bibliographystyle{plainnat}
% replace ``plainnat'' with the style of choice. You can refer to files in the bsts or APA
% subfolder, e.g.
 \bibliographystyle{APA/apa-good}  % or
 \bibliography{thesis}
 % Comment the above two lines and uncomment the next line to use biblatex-chicago.
 %\printbibliography[heading=bibintoc]

% Finally, an index would go here... but it is also optional.
\end{document}
